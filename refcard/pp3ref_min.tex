\documentclass{article}

\usepackage{listings}
\usepackage{mathptmx}
\usepackage[landscape,a4paper,margin=1.5cm,dvips]{geometry}
\usepackage{multicol}

\lstdefinestyle{pp3}{morecomment=[l]\#,numbers=none,numberstyle=\sffamily\scriptsize,
  keywords={filename,set,color,switch,on,off,penalty,objects\_and\_labels,delete,
    add,reposition,delete_labels,add_labels,set_label_text,text,at,along,towards,tics,
  line_width,line_style,penalties},
  basicstyle=\ttfamily, fontadjust, string=[b]{"},
  stringstyle=\texttt,aboveskip=0pt,belowskip=-\medskipamount,escapeinside=`'
%                                ,backgroundcolor=\color[gray]{0.9}
}

\lstdefinestyle{shell}{numbers=none,numberstyle=\sffamily\scriptsize,
  basicstyle=\ttfamily, fontadjust, columns=[l]fullflexible,
  stringstyle=\texttt,aboveskip=0pt,belowskip=-\medskipamount,escapeinside=`',backgroundcolor={}}

\lstset{style=pp3}

\parindent0pt
\parskip\medskipamount
\newskip\subsubskipamount\subsubskipamount\bigskipamount
\newskip\subskipamount\subskipamount1.5\subsubskipamount
\newcommand{\subskip}{\vspace{\subskipamount}\pagebreak[3]}
\newcommand{\subsubskip}{\vspace{\subsubskipamount}\pagebreak[2]}

\setcounter{secnumdepth}{-2}

\raggedbottom
\begin{document}\raggedright\small\setlength{\columnsep}{1.5cm}
\pagestyle{empty}\thispagestyle{empty}

\begin{multicols*}{3}
\title{PP3 reference card}
\author{\texttt{http://pp3.sourceforge.net}}
\date{for version 1.2}
  \maketitle
\section{General structure}

\begin{lstlisting}
# Some introductory comments,
# i.e. what the file is about to do.

`\textrm{[Section I: Parameters.]}'

objects_and_labels

`\textrm{[Section II: Delete/add/modify objects and/or labels]}'
\end{lstlisting}
Both sections may be empty.  If section~II is empty,
\lstinline{objects_and_labels} should be omitted.


\section{Section I: Parameters}

The following top level keywords are allowed here: \lstinline{filename},
\lstinline{set}, \lstinline{switch}, \lstinline{color}, \lstinline{line_width},
\lstinline{line_style}, and \lstinline{penalties}.\subskip

\begin{lstlisting}
filename output orion.tex
\end{lstlisting}
writes the result to the file \verb|orion.tex|.  The file extension must be
\verb|.tex| and must not be omitted.

\begin{lstlisting}
filename stars               stars.dat
filename nebulae             nebulae.dat
filename label_dimensions    labeldimens.dat
filename constellation_lines lines.dat
filename boundaries          boundaries.dat
filename milky_way           milkyway.dat
\end{lstlisting}
This lets PP3 look for the respective data in the given file.  The
filenames must be full paths from the current directory.

\begin{lstlisting}
filename latex_preamble mypreamble.tex
\end{lstlisting}
makes PP3 include the file \verb|mypreamble.tex| with e.\,g.\ font
parameters in every \LaTeX\ run.

\begin{lstlisting}
filename include myproject.pp3
\end{lstlisting}
includes \verb|myproject.pp3| that itself has a normal input script structure,
except for the limitation that it mustn't include yet another file.\subskip

\begin{lstlisting}
set center_rectascension  5.8
set center_declination    0
\end{lstlisting}
This sets the center of the printed map to the celestial coordinates
$(5^{\mathrm h}48',0^\circ)$.

\begin{lstlisting}
set box_width  15
set box_height 15
\end{lstlisting}
sets the dimensions of the printed map in centimetres.  (15~for both is the
default.)

\begin{lstlisting}
set grad_per_cm 4
\end{lstlisting}
sets the scale of the map in degrees per centimetre.  (4~is the default.)\vspace{\bigskipamount}\pagebreak[2]

\begin{lstlisting}
set constellation ORI
\end{lstlisting}
tells PP3 to highlight the boundaries of constellation Orion.  Must be in
uppercase.\vspace{\bigskipamount}\pagebreak[2]

\begin{lstlisting}
set shortest_constellation_line 0.1
\end{lstlisting}
makes PP3 suppress all constellation lines that are shorter than
0.1~centimetres.  (This is the default.)\vspace{\bigskipamount}\pagebreak[2]

\begin{lstlisting}
set label_skip 0.06
\end{lstlisting}
sets the distance between the outer rim of the celestial object and its label
to 0.06~centimetres.  (This is the default.)\vspace{\bigskipamount}\pagebreak[2]

\begin{lstlisting}
set minimal_nebula_radius 0.1
\end{lstlisting}
means that all nebulae that would be smaller than 0.1~centimetres (in radius)
are printed with a radius of \emph{exactly} 0.1~cm.  (This is the
default.)\vspace{\bigskipamount}\pagebreak[2]

\begin{lstlisting}
set faintest_cluster_magnitude 4.0
\end{lstlisting}
suppresses by default all stellar clusters that are fainter than $4.0^{\mathrm
  m}$.  (This is the default.)\vspace{\bigskipamount}\pagebreak[2]

\begin{lstlisting}
set faintest_diffuse_nebula_magnitude 8.0
\end{lstlisting}
suppresses by default all emission and reflection nebulae and all galaxies that
are fainter than $8.0^{\mathrm m}$.  (This is the default.)\vspace{\bigskipamount}\pagebreak[2]

\begin{lstlisting}
set faintest_star_magnitude 7.0
\end{lstlisting}
suppresses by default all stars and all galaxies that are fainter than
$7.0^{\mathrm m}$.  (This is the default.)\vspace{\bigskipamount}\pagebreak[2]

\begin{lstlisting}
set faintest_star_with_label_magnitude 3.7
\end{lstlisting}
means that for stars as bright as $3.7^{\mathrm m}$ or brighter get
automatically generated labels.  (This is the default.)\vspace{\bigskipamount}\pagebreak[2]

\begin{lstlisting}
set faintest_star_disk_magnitude 4.5
\end{lstlisting}
means that all stars that are fainter than $4.5^{\mathrm m}$ are printed as
mere dots instead of small circles.  (This is the default.)\vspace{\bigskipamount}\pagebreak[2]

\begin{lstlisting}
set minimal_star_radius 0.015
\end{lstlisting}
This sets the radius of the dots for all stars fainter than the limit
determined by \lstinline{faintest_star_disk_magnitude} to
0.015~centimetres.  (This is the default.)\vspace{\bigskipamount}\pagebreak[2]

\begin{lstlisting}
set star_scaling 2.0
\end{lstlisting}
makes all star circles two times bigger than normal.  (Default
is~1.0.)\vspace{\bigskipamount}\pagebreak[2]

\begin{lstlisting}
set fontsize 12
\end{lstlisting}
sets the global font size to 12\,pt.  (Default is~10.)\subskip

\begin{lstlisting}
switch milky_way            on
switch grid                 on
switch ecliptic             on
switch boundaries           on
switch constellation_lines  on
switch labels               on
\end{lstlisting}
This switches the display of the Milky Way, the coordinate grid, the ecliptic,
the constellation boundaries, the boundaries of the highlighted constellation,
and all labels on.  This is the default though.\vspace{\bigskipamount}\pagebreak[2]

\begin{lstlisting}
switch eps_output on
\end{lstlisting}
makes PP3 generating an EPS file from the map via a dvips call.\vspace{\bigskipamount}\pagebreak[2]

\begin{lstlisting}
switch pdf_output on
\end{lstlisting}
makes PP3 generating a PDF file from the map by calling dvips and
ps2pdf.\vspace{\bigskipamount}\pagebreak[2]


\begin{lstlisting}
switch colored_stars off
\end{lstlisting}
This disables PP3's default behaviour to try to illustrate the star's
actual colour in the map.  Instead, all stars get the same colour explicitly
defined by ``\lstinline{color stars}''.\subskip

\begin{lstlisting}
color background              0   0   0.4
color grid                    0   0.298 0.447
color ecliptic                1   0   0
color boundaries              0.5 0.5 0
color highlighted_boundaries  1   1   0
color constellation_lines     0   1   0
color milky_way               0   0   1
color nebulae                 1   1   1
\end{lstlisting}
This sets the colours of these map elements to the respective RGB colour.
The colour values range from 0 to~1.  What you see here are the default
values.

\end{multicols*}

\end{document}

%%% Local Variables: 
%%% mode: latex
%%% TeX-master: t
%%% End: 
